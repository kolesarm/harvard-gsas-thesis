\documentclass[11pt]{gsasthesis} % 10,11 and 12pt fonts allowed

%%%%%%%%%%%%%%%% PACKAGES YOU PROBABLY WANT %%%%%%%%%%%%%%%%
% Include packages you want. The gsasthesis style file already includes
% packages "setspace" and "tocbibind".

\usepackage{etex} % extend the number of registers

% GSAS: "all margins should be at least 1 inch."
\usepackage[margin={1.2in}]{geometry}
% If you want asymmetric margins for two-sided documents, use the "twoside"
% option, as in
% \usepackage[top=1in,bottom=1.5in,left=1in,right=1.5in,twoside]{geometry} The
% left and right options become inner and outer margins The default horizontal
% latex margin ratio is 2:3. The default vertical top:bottom margin ratio is 2:3
% also. You can also set it directly by passing the hmarginratio option to the
% geometry package, as in
% \usepackage[top=1in,left=1in,vmarginratio=2:3,hmarginratio=2:5,twoside]{geometry}

% Appendix package. Not necessary, but it does make managing appendices easier
\usepackage[titletoc]{appendix}

%%%%%%%%%%%%%%%% PACKAGES MAY WANT %%%%%%%%%%%%%%%%

% sideways tables and figures
\usepackage{rotating}

% tables that spill over multiple pages
\usepackage{longtable}

% references
\usepackage{natbib}

% fonts that are nicer than defaults
\usepackage[sc]{mathpazo}
\usepackage{courier}

% Use 8-bit encoding that has 256 glyphs, pretty please
\usepackage[utf8]{inputenc}
\usepackage[T1]{fontenc}

% babel is required for blindtext, which generates random text
\usepackage[english]{babel}
\usepackage{blindtext}

% math support
\usepackage{amsmath}

% Slightly tweak font spacing for aesthetics
\usepackage{microtype}

% You need the footmisc package with the stable option if you want to have
% footnotes inside section titles, for example to say that a particular chapter
% has been co-authored with someone. The multiple option ensures that there is a
% comma between two consecutive footnotes
\usepackage[stable,multiple]{footmisc}


% Nicer captions
\RequirePackage[font=small,format=plain,labelfont=bf,textfont=it]{caption}
\addtolength{\abovecaptionskip}{1ex}
\addtolength{\belowcaptionskip}{1ex}


%%%%%%%%%%%%%%%% COMPULSORY FIELDS %%%%%%%%%%%%%%%%

\title{Essays on Thesis-formatting} % needs to match title on DAC
\author{Econ Gradstudent} % full name as it appears on your GSAS record, needs
                          % to match name on DAC
\degreename{Doctor of Philosophy}
\degreefield{Thesis-formatting} % Official name of subject as listed in GSAS
                                % handbook
\department{The Department of Economics} % official name of department
\degreemonth{April} % Month of Defense (i.e. month when DAC was signed)
\degreeyear{2013} % Year the DAC was signed
\principaladvisor{Professor John Principal}

% Optionally, you can add a second advisor, but you can't have three
\secondadvisor{Professor George Secondary}



\begin{document}

%%%%%%%%%%%%%%%% FRONTMATTER %%%%%%%%%%%%%%%%

\pagenumbering{roman} % GSAS wants roman page numbers for frontmatter

% the following four pages are required in that order. The first two pages are
% not allowed to have page numbers, this is taken care of in the class file.
\thesistitlepage
\copyrightpage
\begin{abstract}
  An abstract should be less than 350 words. Here's some filler text. \blindtext
\end{abstract}

% Center headings for table of contents, LOT, and LOF and make them smaller so
% that "Abstract", "Acknowledgments" and "Contents" all look alike. Comment out
% if you want the default. If you want more control, use the "tocloft" package.
\renewcommand{\contentsname}{\protect\centering\protect\Large Contents}
\renewcommand{\listtablename}{\protect\centering\protect\Large List of Tables}
\renewcommand{\listfigurename}{\protect\centering\protect\Large List of Figures}

\tableofcontents % Table of contents

% The rest of the front matter: Lists of tables, figures, dedication and
% acknowledment is optional. Comment out whatever you don't like
\listoftables
\listoffigures
\begin{acknowledgments}
  \blindtext
\end{acknowledgments}
\begin{dedication}
  To my parents
\end{dedication}


%%%%%%%%%%%%%%%% MAIN BODY %%%%%%%%%%%%%%%%
\pagenumbering{arabic} % reset page numbering and switch to arabic

% Introductory chapter. Comment out if you don't have an intro chapter, but I
% think most committees expect you to have one.
% Don't number the intro chapter, but add to to the table of contents
\addcontentsline{toc}{chapter}{Introduction}
\chapter*{Introduction}\label{ch:intro}
Introductiory chapter that talks about all three papers for a little bit longer
than the abstract.

%%% Local Variables:
%%% mode: latex
%%% TeX-master: t
%%% End:



\chapter{Hook\footnote{Co-authored with my advisor}}\label{ch:1}
\section{Introduction}
\label{section_intro}

Block Quotations are automatically single spaced. Here's a dummy quotation:
\begin{quotation}
  \blindtext
\end{quotation}

\section{Motivating Example}
Table \ref{tab:label} shows stuff. \blindtext
\begin{table}[tp]
  \centering
  \caption[Table heading]{Table heading goes on top of the table}
  \label{tab:label}
  \renewcommand{\arraystretch}{1.2} % single-spacing
  \begin{tabular}{@{}lll@{}}
    Tables & should \\
    Be & double\\
    spaced & unless & \\
    they are & long\\
    This & table\\
    is & getting\\
    long\\
    so & I\\
    manually\\
    changed &it \\
    to & single\\
    spacing
  \end{tabular}
\end{table}
Table \ref{tab:label2} shows stuff also.
\begin{table}[tp]
  \centering
  \caption{Use consistent format for captions}
  \label{tab:label2}
  \begin{tabular}{@{}lllll@{}}
    Table & should & be & placed\\
    within & text, & as & close\\
    to & its first mention\\
    as & possible. & Not at the end\\
    of a chapter & or dissertation
  \end{tabular}
\end{table}

\blindtext[2]


\chapter{Line\footnote{Co-authored with my other advisor}}\label{ch:2}
\section{Introduction}
\blindmathtrue\blindtext

\section{Potential outcomes framework}
\label{sec:potent-outc-fram}
\blindmathtrue\blindtext.\footnote{Footnotes are single-spaced. \blindtext}

\section{Conclusion}
I conclude that:
\blinditemize


\chapter{Sinker}\label{ch:3}
\section{Introduction}
Some people just cite papers in introductions for no reason.
\citet{ar49,pearson01,spe04}.

\section{Setup}
\label{sec:potent-outc-fram}
\blindmathtrue\blindtext\ See Figure \ref{fig:figure1} for illustration.
\begin{figure}[ht]
  \centering
\begin{verbatim}
#include <iostream>
int main(int argc, char** argv) {
  std::cout << "Hello World." << std::endl;
  return 0;
}
\end{verbatim}
  \caption{Captions for figures go at the bottom of the figure.}
  \label{fig:figure1}
\end{figure}

\section{Conclusion}
\blindmathfalse\blindtext[2]


%%%%%%%%%%%%%%%% BACK MATTER %%%%%%%%%%%%%%%%

% Put appendices, bibliography, and supplemental materials here

% The bibliography may be single spaced within each entry, but must be
% double-spaced between each entry. Most bibliography styles leave space between
% entries, so that shouldn't be a problem.
\begin{singlespacing}
  % I like "References" better than "Bibliography"
  \renewcommand{\bibname}{References}

  % Any bibliohgraphy style that leaves space between entries is fine
  \bibliographystyle{ecca}
  \bibliography{references}
\end{singlespacing}

% Appendices from all chapters should go at the end
\begin{appendices}

\chapter{Appendix to Chapter \ref{ch:1}}\label{cha:append-chapt-refch:1}

\section{Auxiliary Lemmata}
Fundamental identity
\begin{equation}
  \label{eq:A}
  e^{i\pi}=-1.
\end{equation}
Equivalence relation
\begin{equation}
  \label{eq:B}
  A=B.
\end{equation}

\section{Proofs}

\chapter{Appendix to Chapter \ref{ch:3}}

\section{Proofs}

\section{Supplementary Figures}

\begin{figure}[ht]
  \centering Supplementary figures and tables should be placed in the appendix,
  not at the end ofa chapter
  \caption{Supplementary Figure}
  \label{fig:figuresup1}
\end{figure}

\begin{figure}[ht]
  \centering
  This is another supplementary figure.
  \caption{Another Figure}
  \label{fig:figuresup3}
\end{figure}


\end{appendices}


\end{document}
